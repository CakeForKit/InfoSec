\chapter{Теоретическая часть}

Виды симметричного шифрования:
\begin{enumerate}[label={\arabic*)}]
	\item Поточные: Шифруют данные побитово/побайтово (RC4);
	\item Блочные: Шифруют данные блоками фиксированного размера (DES, AES);	
\end{enumerate}

Алгоритмы перестановки — это методы, которые изменяют порядок следования элементов (но не их значения) посредством присваивания и перестановки их значений. Пример: IP в DES.

Алгоритмы подстановки — это методы шифрования, в которых элементы исходного открытого текста заменяются зашифрованным текстом в соответствии с некоторым правилом. Пример: шифр Цезаря.

Алгоритм DES использует методы перестановки и подстановки.

\chapter{Описание алгоритма симметричного шифрования (AЕS)}

%На рисунке~\ref{img:img/genKeys} приведена схема генерации подключей, которая выполняется перед началом шифрования по алгоритму DES.
%\FloatBarrier
%\imgw{0.6\textwidth}{img/genKeys}{Схема генерации генерации подключей}
%\FloatBarrier
%\clearpage

На рисунке~\ref{img:img/aes} приведена схема алгоритма симметричного шифрования (AЕS).

\FloatBarrier
\imgw{0.6\textwidth}{img/aes}{Схема алгоритма симметричного шифрования (AЕS)}
\FloatBarrier



\chapter{Пример работы алгоритма симметричного шифрования (AЕS)}

На рисунке~\ref{img:img/example} приведен пример работы алгоритма симметричного шифрования (AЕS).

\FloatBarrier
\imgw{1\textwidth}{img/example}{Пример работы алгоритма симметричного шифрования (AЕS)}
\FloatBarrier

\clearpage
На рисунках~\ref{img:img/text1}-\ref{img:img/text3} приведено содержимое исходного, зашифрованного и расшифрованного фалов.

\FloatBarrier
\imgw{1\textwidth}{img/text1}{Содержимое исходного и зашифрованного файлов}
\FloatBarrier
\imgw{1\textwidth}{img/text2}{Содержимое расшифрованного файла}
\FloatBarrier

\clearpage
На рисунке~\ref{img:img/text3} приведен пример шифрования и расшифровки при попытке расшифровки файла сторонним ключом.
\FloatBarrier
\imgw{1\textwidth}{img/text3}{Пример шифрования и расшифровки при попытке расшифровки файла сторонним ключом}
\FloatBarrier

\chapter{Реализация алгоритма симметричного шифрования (AЕS)}
В качестве средства реализации алгоритма симметричного шифрования (AЕS) был выбран язык Go.

\begin{lstlisting}[style=golang, caption={Реализация алгоритма симметричного шифрования (AЕS)}, label=lst:codegolang]
package main

import (
"crypto/rand"
"errors"
"fmt"
"io"
"os"
"strconv"
)

const (
AESBlockSize   = 16
AESStateDim    = 4
AESRounds128   = 10
AESRounds192   = 12
AESRounds256   = 14
BitsPerByte    = 8
WordSize       = 4
GFReducingPoly = 0x1B
GFMSBMask      = 0x80
)

type AESKeySize int

const (
AESKeySize128 AESKeySize = 16
AESKeySize192 AESKeySize = 24
AESKeySize256 AESKeySize = 32
)

type AESError int

const (
AESSuccess AESError = iota
AESErrorUnsupportedKeySize
AESErrorMemoryAllocation
AESErrorInvalidInput
)

func (e AESError) String() string {
	switch e {
		case AESSuccess:
		return "Success"
		case AESErrorUnsupportedKeySize:
		return "Unsupported key size"
		case AESErrorMemoryAllocation:
		return "Memory allocation failed"
		case AESErrorInvalidInput:
		return "Invalid input parameters"
		default:
		return "Unknown error"
	}
}

var sbox = [256]byte{
	0x63, 0x7c, 0x77, 0x7b, 0xf2, 0x6b, 0x6f, 0xc5, 0x30, 0x01, 0x67, 0x2b, 0xfe, 0xd7, 0xab, 0x76,
	0xca, 0x82, 0xc9, 0x7d, 0xfa, 0x59, 0x47, 0xf0, 0xad, 0xd4, 0xa2, 0xaf, 0x9c, 0xa4, 0x72, 0xc0,
	0xb7, 0xfd, 0x93, 0x26, 0x36, 0x3f, 0xf7, 0xcc, 0x34, 0xa5, 0xe5, 0xf1, 0x71, 0xd8, 0x31, 0x15,
	0x04, 0xc7, 0x23, 0xc3, 0x18, 0x96, 0x05, 0x9a, 0x07, 0x12, 0x80, 0xe2, 0xeb, 0x27, 0xb2, 0x75,
	0x09, 0x83, 0x2c, 0x1a, 0x1b, 0x6e, 0x5a, 0xa0, 0x52, 0x3b, 0xd6, 0xb3, 0x29, 0xe3, 0x2f, 0x84,
	0x53, 0xd1, 0x00, 0xed, 0x20, 0xfc, 0xb1, 0x5b, 0x6a, 0xcb, 0xbe, 0x39, 0x4a, 0x4c, 0x58, 0xcf,
	0xd0, 0xef, 0xaa, 0xfb, 0x43, 0x4d, 0x33, 0x85, 0x45, 0xf9, 0x02, 0x7f, 0x50, 0x3c, 0x9f, 0xa8,
	0x51, 0xa3, 0x40, 0x8f, 0x92, 0x9d, 0x38, 0xf5, 0xbc, 0xb6, 0xda, 0x21, 0x10, 0xff, 0xf3, 0xd2,
	0xcd, 0x0c, 0x13, 0xec, 0x5f, 0x97, 0x44, 0x17, 0xc4, 0xa7, 0x7e, 0x3d, 0x64, 0x5d, 0x19, 0x73,
	0x60, 0x81, 0x4f, 0xdc, 0x22, 0x2a, 0x90, 0x88, 0x46, 0xee, 0xb8, 0x14, 0xde, 0x5e, 0x0b, 0xdb,
	0xe0, 0x32, 0x3a, 0x0a, 0x49, 0x06, 0x24, 0x5c, 0xc2, 0xd3, 0xac, 0x62, 0x91, 0x95, 0xe4, 0x79,
	0xe7, 0xc8, 0x37, 0x6d, 0x8d, 0xd5, 0x4e, 0xa9, 0x6c, 0x56, 0xf4, 0xea, 0x65, 0x7a, 0xae, 0x08,
	0xba, 0x78, 0x25, 0x2e, 0x1c, 0xa6, 0xb4, 0xc6, 0xe8, 0xdd, 0x74, 0x1f, 0x4b, 0xbd, 0x8b, 0x8a,
	0x70, 0x3e, 0xb5, 0x66, 0x48, 0x03, 0xf6, 0x0e, 0x61, 0x35, 0x57, 0xb9, 0x86, 0xc1, 0x1d, 0x9e,
	0xe1, 0xf8, 0x98, 0x11, 0x69, 0xd9, 0x8e, 0x94, 0x9b, 0x1e, 0x87, 0xe9, 0xce, 0x55, 0x28, 0xdf,
	0x8c, 0xa1, 0x89, 0x0d, 0xbf, 0xe6, 0x42, 0x68, 0x41, 0x99, 0x2d, 0x0f, 0xb0, 0x54, 0xbb, 0x16,
}

var rsbox = [256]byte{
	0x52, 0x09, 0x6a, 0xd5, 0x30, 0x36, 0xa5, 0x38, 0xbf, 0x40, 0xa3, 0x9e, 0x81, 0xf3, 0xd7, 0xfb,
	0x7c, 0xe3, 0x39, 0x82, 0x9b, 0x2f, 0xff, 0x87, 0x34, 0x8e, 0x43, 0x44, 0xc4, 0xde, 0xe9, 0xcb,
	0x54, 0x7b, 0x94, 0x32, 0xa6, 0xc2, 0x23, 0x3d, 0xee, 0x4c, 0x95, 0x0b, 0x42, 0xfa, 0xc3, 0x4e,
	0x08, 0x2e, 0xa1, 0x66, 0x28, 0xd9, 0x24, 0xb2, 0x76, 0x5b, 0xa2, 0x49, 0x6d, 0x8b, 0xd1, 0x25,
	0x72, 0xf8, 0xf6, 0x64, 0x86, 0x68, 0x98, 0x16, 0xd4, 0xa4, 0x5c, 0xcc, 0x5d, 0x65, 0xb6, 0x92,
	0x6c, 0x70, 0x48, 0x50, 0xfd, 0xed, 0xb9, 0xda, 0x5e, 0x15, 0x46, 0x57, 0xa7, 0x8d, 0x9d, 0x84,
	0x90, 0xd8, 0xab, 0x00, 0x8c, 0xbc, 0xd3, 0x0a, 0xf7, 0xe4, 0x58, 0x05, 0xb8, 0xb3, 0x45, 0x06,
	0xd0, 0x2c, 0x1e, 0x8f, 0xca, 0x3f, 0x0f, 0x02, 0xc1, 0xaf, 0xbd, 0x03, 0x01, 0x13, 0x8a, 0x6b,
	0x3a, 0x91, 0x11, 0x41, 0x4f, 0x67, 0xdc, 0xea, 0x97, 0xf2, 0xcf, 0xce, 0xf0, 0xb4, 0xe6, 0x73,
	0x96, 0xac, 0x74, 0x22, 0xe7, 0xad, 0x35, 0x85, 0xe2, 0xf9, 0x37, 0xe8, 0x1c, 0x75, 0xdf, 0x6e,
	0x47, 0xf1, 0x1a, 0x71, 0x1d, 0x29, 0xc5, 0x89, 0x6f, 0xb7, 0x62, 0x0e, 0xaa, 0x18, 0xbe, 0x1b,
	0xfc, 0x56, 0x3e, 0x4b, 0xc6, 0xd2, 0x79, 0x20, 0x9a, 0xdb, 0xc0, 0xfe, 0x78, 0xcd, 0x5a, 0xf4,
	0x1f, 0xdd, 0xa8, 0x33, 0x88, 0x07, 0xc7, 0x31, 0xb1, 0x12, 0x10, 0x59, 0x27, 0x80, 0xec, 0x5f,
	0x60, 0x51, 0x7f, 0xa9, 0x19, 0xb5, 0x4a, 0x0d, 0x2d, 0xe5, 0x7a, 0x9f, 0x93, 0xc9, 0x9c, 0xef,
	0xa0, 0xe0, 0x3b, 0x4d, 0xae, 0x2a, 0xf5, 0xb0, 0xc8, 0xeb, 0xbb, 0x3c, 0x83, 0x53, 0x99, 0x61,
	0x17, 0x2b, 0x04, 0x7e, 0xba, 0x77, 0xd6, 0x26, 0xe1, 0x69, 0x14, 0x63, 0x55, 0x21, 0x0c, 0x7d,
}

var rcon = [32]byte{
	0x00, 0x01, 0x02, 0x04, 0x08, 0x10, 0x20, 0x40, 0x80, 0x1b, 0x36,
	0x6c, 0xd8, 0xab, 0x4d, 0x9a, 0x2f, 0x5e, 0xbc, 0x63, 0xc6, 0x97,
	0x35, 0x6a, 0xd4, 0xb3, 0x7d, 0xfa, 0xef, 0xc5, 0x91, 0x39,
}

type AESState [AESStateDim][AESStateDim]byte

func secureZeroMemory(data []byte) {
	for i := range data {
		data[i] = 0
	}
}

func wordRotateLeft(word []byte) {
	if len(word) != WordSize {
		return
	}
	temp := word[0]
	word[0] = word[1]
	word[1] = word[2]
	word[2] = word[3]
	word[3] = temp
}

func keyScheduleCore(word []byte, iteration byte) {
	wordRotateLeft(word)
	for i := 0; i < WordSize; i++ {
		word[i] = sbox[word[i]]
	}
	word[0] ^= rcon[iteration]
}

func aesExpandKey(key []byte, keySize AESKeySize) ([]byte, error) {
	var numRounds int
	switch keySize {
		case AESKeySize128:
		numRounds = AESRounds128
		case AESKeySize192:
		numRounds = AESRounds192
		case AESKeySize256:
		numRounds = AESRounds256
		default:
		return nil, errors.New("unsupported key size")
	}
	
	expandedKeySize := AESBlockSize * (numRounds + 1)
	expandedKey := make([]byte, expandedKeySize)
	copy(expandedKey, key)
	
	currentSize := len(key)
	rconIteration := byte(1)
	tempWord := make([]byte, WordSize)
	
	for currentSize < expandedKeySize {
		copy(tempWord, expandedKey[currentSize-WordSize:currentSize])
		
		if currentSize%int(keySize) == 0 {
			keyScheduleCore(tempWord, rconIteration)
			rconIteration++
		}
		
		if keySize == AESKeySize256 && currentSize%int(keySize) == AESBlockSize {
			for i := 0; i < WordSize; i++ {
				tempWord[i] = sbox[tempWord[i]]
			}
		}
		
		for i := 0; i < WordSize; i++ {
			expandedKey[currentSize] = expandedKey[currentSize-int(keySize)] ^ tempWord[i]
			currentSize++
		}
	}
	
	return expandedKey, nil
}

func subBytes(state *AESState) {
	for r := 0; r < AESStateDim; r++ {
		for c := 0; c < AESStateDim; c++ {
			state[r][c] = sbox[state[r][c]]
		}
	}
}

func invSubBytes(state *AESState) {
	for r := 0; r < AESStateDim; r++ {
		for c := 0; c < AESStateDim; c++ {
			state[r][c] = rsbox[state[r][c]]
		}
	}
}

func shiftRows(state *AESState) {
	// Row 1: 1-byte left shift
	temp := state[1][0]
	state[1][0] = state[1][1]
	state[1][1] = state[1][2]
	state[1][2] = state[1][3]
	state[1][3] = temp
	
	// Row 2: 2-byte left shift
	temp = state[2][0]
	state[2][0] = state[2][2]
	state[2][2] = temp
	temp = state[2][1]
	state[2][1] = state[2][3]
	state[2][3] = temp
	
	// Row 3: 3-byte left shift
	temp = state[3][0]
	state[3][0] = state[3][3]
	state[3][3] = state[3][2]
	state[3][2] = state[3][1]
	state[3][1] = temp
}

func invShiftRows(state *AESState) {
	// Row 1: 1-byte right shift
	temp := state[1][3]
	state[1][3] = state[1][2]
	state[1][2] = state[1][1]
	state[1][1] = state[1][0]
	state[1][0] = temp
	
	// Row 2: 2-byte right shift
	temp = state[2][0]
	state[2][0] = state[2][2]
	state[2][2] = temp
	temp = state[2][1]
	state[2][1] = state[2][3]
	state[2][3] = temp
	
	// Row 3: 3-byte right shift
	temp = state[3][3]
	state[3][3] = state[3][0]
	state[3][0] = state[3][1]
	state[3][1] = state[3][2]
	state[3][2] = temp
}

func galoisMul(a, b byte) byte {
	p := byte(0)
	for i := 0; i < BitsPerByte; i++ {
		if b&1 != 0 {
			p ^= a
		}
		
		hiBitSet := a & GFMSBMask
		a <<= 1
		
		if hiBitSet != 0 {
			a ^= GFReducingPoly
		}
		
		b >>= 1
	}
	return p
}

func mixColumns(state *AESState) {
	var t [AESStateDim]byte
	for c := 0; c < AESStateDim; c++ {
		for r := 0; r < AESStateDim; r++ {
			t[r] = state[r][c]
		}
		
		state[0][c] = galoisMul(t[0], 2) ^ galoisMul(t[1], 3) ^ t[2] ^ t[3]
		state[1][c] = t[0] ^ galoisMul(t[1], 2) ^ galoisMul(t[2], 3) ^ t[3]
		state[2][c] = t[0] ^ t[1] ^ galoisMul(t[2], 2) ^ galoisMul(t[3], 3)
		state[3][c] = galoisMul(t[0], 3) ^ t[1] ^ t[2] ^ galoisMul(t[3], 2)
	}
}

func invMixColumns(state *AESState) {
	var t [AESStateDim]byte
	for c := 0; c < AESStateDim; c++ {
		for r := 0; r < AESStateDim; r++ {
			t[r] = state[r][c]
		}
		
		state[0][c] = galoisMul(t[0], 14) ^ galoisMul(t[1], 11) ^ galoisMul(t[2], 13) ^ galoisMul(t[3], 9)
		state[1][c] = galoisMul(t[0], 9) ^ galoisMul(t[1], 14) ^ galoisMul(t[2], 11) ^ galoisMul(t[3], 13)
		state[2][c] = galoisMul(t[0], 13) ^ galoisMul(t[1], 9) ^ galoisMul(t[2], 14) ^ galoisMul(t[3], 11)
		state[3][c] = galoisMul(t[0], 11) ^ galoisMul(t[1], 13) ^ galoisMul(t[2], 9) ^ galoisMul(t[3], 14)
	}
}

func addRoundKey(state *AESState, roundKey []byte) {
	for c := 0; c < AESStateDim; c++ {
		for r := 0; r < AESStateDim; r++ {
			state[r][c] ^= roundKey[c*AESStateDim+r]
		}
	}
}

func aesEncryptBlock(plaintext []byte, key []byte, keySize AESKeySize) ([]byte, error) {
	if len(plaintext) != AESBlockSize {
		return nil, errors.New("plaintext must be 16 bytes")
	}
	
	var numRounds int
	switch keySize {
		case AESKeySize128:
		numRounds = AESRounds128
		case AESKeySize192:
		numRounds = AESRounds192
		case AESKeySize256:
		numRounds = AESRounds256
		default:
		return nil, errors.New("unsupported key size")
	}
	
	expandedKey, err := aesExpandKey(key, keySize)
	if err != nil {
		return nil, err
	}
	defer secureZeroMemory(expandedKey)
	
	var state AESState
	for r := 0; r < AESStateDim; r++ {
		for c := 0; c < AESStateDim; c++ {
			state[r][c] = plaintext[r+AESStateDim*c]
		}
	}
	
	addRoundKey(&state, expandedKey)
	for round := 1; round < numRounds; round++ {
		subBytes(&state)
		shiftRows(&state)
		mixColumns(&state)
		addRoundKey(&state, expandedKey[AESBlockSize*round:])
	}
	subBytes(&state)
	shiftRows(&state)
	addRoundKey(&state, expandedKey[AESBlockSize*numRounds:])
	
	ciphertext := make([]byte, AESBlockSize)
	for r := 0; r < AESStateDim; r++ {
		for c := 0; c < AESStateDim; c++ {
			ciphertext[r+AESStateDim*c] = state[r][c]
		}
	}
	
	return ciphertext, nil
}

func aesDecryptBlock(ciphertext []byte, key []byte, keySize AESKeySize) ([]byte, error) {
	if len(ciphertext) != AESBlockSize {
		return nil, errors.New("ciphertext must be 16 bytes")
	}
	
	var numRounds int
	switch keySize {
		case AESKeySize128:
		numRounds = AESRounds128
		case AESKeySize192:
		numRounds = AESRounds192
		case AESKeySize256:
		numRounds = AESRounds256
		default:
		return nil, errors.New("unsupported key size")
	}
	
	expandedKey, err := aesExpandKey(key, keySize)
	if err != nil {
		return nil, err
	}
	defer secureZeroMemory(expandedKey)
	
	var state AESState
	for r := 0; r < AESStateDim; r++ {
		for c := 0; c < AESStateDim; c++ {
			state[r][c] = ciphertext[r+AESStateDim*c]
		}
	}
	
	addRoundKey(&state, expandedKey[AESBlockSize*numRounds:])
	for round := numRounds; round > 1; round-- {
		invShiftRows(&state)
		invSubBytes(&state)
		addRoundKey(&state, expandedKey[AESBlockSize*(round-1):])
		invMixColumns(&state)
	}
	invShiftRows(&state)
	invSubBytes(&state)
	addRoundKey(&state, expandedKey)
	
	plaintext := make([]byte, AESBlockSize)
	for r := 0; r < AESStateDim; r++ {
		for c := 0; c < AESStateDim; c++ {
			plaintext[r+AESStateDim*c] = state[r][c]
		}
	}
	
	return plaintext, nil
}

func encryptFile(inputPath, outputPath string, key []byte, keySize AESKeySize) error {
	inputFile, err := os.Open(inputPath)
	if err != nil {
		return err
	}
	defer inputFile.Close()
	
	outputFile, err := os.Create(outputPath)
	if err != nil {
		return err
	}
	defer outputFile.Close()
	
	// IV (Initialization Vector)
	iv := make([]byte, AESBlockSize)
	if _, err := rand.Read(iv); err != nil {
		return err
	}
	
	if _, err := outputFile.Write(iv); err != nil {
		return err
	}
	
	buffer := make([]byte, AESBlockSize)
	previousBlock := make([]byte, AESBlockSize)
	copy(previousBlock, iv)
	
	for {
		n, err := inputFile.Read(buffer)
		if err != nil && err != io.EOF {
			return err
		}
		
		if n == 0 {
			break
		}
		
		if n < AESBlockSize {
			padding := byte(AESBlockSize - n)
			for i := n; i < AESBlockSize; i++ {
				buffer[i] = padding
			}
		}
		
		for i := 0; i < AESBlockSize; i++ {
			buffer[i] ^= previousBlock[i]
		}
		
		encryptedBlock, err := aesEncryptBlock(buffer, key, keySize)
		if err != nil {
			return err
		}
		
		if _, err := outputFile.Write(encryptedBlock); err != nil {
			return err
		}
		
		copy(previousBlock, encryptedBlock)
	}
	
	return nil
}

func decryptFile(inputPath, outputPath string, key []byte, keySize AESKeySize) error {
	inputFile, err := os.Open(inputPath)
	if err != nil {
		return err
	}
	defer inputFile.Close()
	
	outputFile, err := os.Create(outputPath)
	if err != nil {
		return err
	}
	defer outputFile.Close()
	
	iv := make([]byte, AESBlockSize)
	if _, err := inputFile.Read(iv); err != nil {
		return err
	}
	
	fileInfo, err := inputFile.Stat()
	if err != nil {
		return err
	}
	
	fileSize := fileInfo.Size()
	if fileSize%AESBlockSize != 0 || fileSize < int64(AESBlockSize) {
		return errors.New("invalid encrypted file size")
	}
	
	previousBlock := make([]byte, AESBlockSize)
	copy(previousBlock, iv)
	buffer := make([]byte, AESBlockSize)
	
	for {
		n, err := inputFile.Read(buffer)
		if err != nil && err != io.EOF {
			return err
		}
		
		if n == 0 {
			break
		}
		
		decryptedBlock, err := aesDecryptBlock(buffer, key, keySize)
		if err != nil {
			return err
		}
		
		for i := 0; i < AESBlockSize; i++ {
			decryptedBlock[i] ^= previousBlock[i]
		}
		
		copy(previousBlock, buffer)
		
		bytesToWrite := AESBlockSize
		if inputFile, err := inputFile.Seek(0, io.SeekCurrent); err == nil {
			if inputFile == fileSize {
				padding := decryptedBlock[AESBlockSize-1]
				if padding > 0 && padding <= AESBlockSize {
					bytesToWrite = AESBlockSize - int(padding)
				}
			}
		}
		
		if _, err := outputFile.Write(decryptedBlock[:bytesToWrite]); err != nil {
			return err
		}
	}
	
	return nil
}

func generateKey(keySize AESKeySize) ([]byte, error) {
	key := make([]byte, int(keySize))
	if _, err := rand.Read(key); err != nil {
		return nil, err
	}
	return key, nil
}

func saveKeyToFile(key []byte, filename string) error {
	return os.WriteFile(filename, key, 0600)
}

func loadKeyFromFile(filename string) ([]byte, error) {
	key, err := os.ReadFile(filename)
	if err != nil {
		return nil, err
	}
	
	switch len(key) {
		case 16, 24, 32:
		return key, nil
		default:
		return nil, errors.New("invalid key size. Must be 16, 24, or 32 bytes")
	}
}

\end{lstlisting}
