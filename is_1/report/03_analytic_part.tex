\chapter{Теоретическая часть}

Информация -- это сведения (сообщения, данные) независимо от формы их представления, воспринимаемые человеком или специальными устройствами как отражение фактов материального или духовного мира в процессе коммуникации.

Защита информации -- это комплекс мероприятий, направленных на обеспечение конфиденциальности (недоступности информации для неавторизованных лиц), целостности (точности и полноты информации) и доступности (возможности получить информацию авторизованным пользователям) информации. Это процесс противодействия угрозам безопасности информации.

Актив -- это любой компонент информационной системы (данные, оборудование, программное обеспечение, персонал, услуги, репутация), который имеет ценность для организации и поэтому требует защиты.

Информационная сфера -- это совокупность информации, информационной инфраструктуры, субъектов, осуществляющих сбор, формирование, распространение и использование информации, а также системы регулирования возникающих при этом общественных отношений. Это среда, в которой существует и циркулирует информация.

Угроза -- это потенциальная возможность того, что определенное лицо, действие, событие или явление (источник угрозы) преднамеренно или случайно нарушит безопасность информации (ее конфиденциальность, целостность, доступность), нанеся ущерб владельцу или пользователю информации.

Шифровальная машина «Энигма» -- это портативная электромеханическая шифровальная машина, использовавшаяся в XX веке (в основном нацистской Германией во Второй мировой войне) для защиты служебной переписки. Ее основным принципом работы было многоалфавитное шифрование с изменяющимся алфавитом замены после каждой буквы, реализуемое с помощью системы вращающихся роторов.

Одноалфавитная подстановка -- подстановка, при которой каждый символ открытого текста заменяется на один и тот же символ шифрованного текста на протяжении всего сообщения.

Многоалфавитная подстановка -- подстановка, при которой каждый символ открытого текста может заменяться на разные символы шифрованного текста в зависимости от своей позиции в тексте и используемого алфавита замены.

Алгоритм Энигма относится к многоалфавитным подстановочным шифрам.


\chapter{Описание алгоритма работы электронного аналога шифровальной машины «Энигма»}

На рисунке~\ref{img:img/enigma_algo} приведена схема работы шифровальной машины «Энигма»

\FloatBarrier
\imgw{0.8\textwidth}{img/enigma_algo}{Cхема работы шифровальной машины «Энигма»}
\FloatBarrier

\chapter{Реализация электронного аналога шифровальной машины «Энигма»}
\begin{lstlisting}[style=golang, caption={Реализация электронного аналога шифровальной машины «Энигма»}, label=lst:codegolang]
type Enigma interface {
	EncryptAlpha(alpha byte) byte
	EncryptText(text []byte) []byte
	SetRotorPositions(poses []byte) error
}

type enigma struct {
	switchingPanel Rotor
	rotors         []Rotor
	reflector      Reflector
}

func NewEnigma(switchingPanel Rotor, rotors []Rotor, reflector Reflector) Enigma {
	return &enigma{
		switchingPanel: switchingPanel,
		rotors:         rotors,
		reflector:      reflector,
	}
}

func (e *enigma) EncryptText(text []byte) []byte {
	resText := make([]byte, len(text))
	for i, v := range text {
		resText[i] = e.EncryptAlpha(v)
	}
	return resText
}

func (e *enigma) EncryptAlpha(alpha byte) byte {
	alpha = e.switchingPanel.SwitchTo(alpha)
	Nrotors := len(e.rotors)
	e.rotors[0].Rotate()
	nextA := e.rotors[0].Transform(alpha, 0)
	lastRing := e.rotors[0].GetRing()
	
	for i := 1; i < Nrotors; i++ {
		if e.rotors[i-1].GetRing() == e.rotors[i-1].GetSteppingPos() {
			e.rotors[i].Rotate()
		}
		nextA = e.rotors[i].Transform(nextA, lastRing)
		lastRing = e.rotors[i].GetRing()
	}
	
	nextA = e.reflector.Transform(nextA, lastRing, -1)
	
	lastRing = 0
	for i := len(e.rotors) - 1; i >= 0; i-- {
		nextA = e.rotors[i].TransformBack(nextA, lastRing)
		lastRing = e.rotors[i].GetRing()
	}
	nextA = byte((int(nextA) - int(lastRing) + alphabetSize) % alphabetSize)
	nextA = e.switchingPanel.SwitchFrom(nextA)
	
	return nextA
}

func (e *enigma) SetRotorPositions(poses []byte) error {
	if len(poses) != len(e.rotors) {
		return ErrLenPoses
	}
	for i := 0; i < len(e.rotors); i++ {
		e.rotors[i].SetRing(poses[i])
	}
	return nil
}
\end{lstlisting}


%
%
%В данном разделе описывается анализ предметной области базы данных музея, формализация задачи и выбор базы данных по модели хранения.
%
%\section{Анализ предметной области}
%
%
%В данной работе под словом музей понимается учреждение, занимающееся хранением и выставлением на обозрение произведений искусства. Произведением искусства может считаться любой объект, имеющий эстетическую и историческую ценность. Поэтому разные музеи хранят разного рода объекты и на сайтах называют их по разному, например, экспонаты, шедевры, картины, произведения, объекты.
%Для создания базы данных музея были рассмотрены аналоги, которые представлены в виде информации на сайте Третьяковской галереи, Пушкинского музея, Эрмитажа и Лувра.
%
%\subsection{Третьяковская галерея}
%
%\subsection{Сравнение аналогов}
%Для сравнения описанных выше аналогов хранилищ произведений искусства были выбраны критерии:
%\begin{enumerate}[label={\arabic*)}]
%	\item характеристики произведений искусства представленные на сайте;
%	\item параметры сортировки экспонатов;	
%	\item параметры фильтрации экспонатов;	
%	\item наличие информации о местоположении картин;	
%	\item наличие информации о предстоящих выставках;	
%%	\item наличие выпадающего списка в фильтрах;	
%%	\item возможность увидеть полную информацию о произведении искусства;
%	\item наличие информации о текущих выставках;
%	\item возможность приобретения билетов на выставки;
%%	\item возможность спонсорства (программа “Друг музея”);
%\end{enumerate}
%
%\begin{longtable}{|
%		>{\centering\arraybackslash}m{.2\textwidth - 2\tabcolsep}|
%		>{\centering\arraybackslash}m{.2\textwidth - 2\tabcolsep}|
%		>{\centering\arraybackslash}m{.2\textwidth - 2\tabcolsep}|
%		>{\centering\arraybackslash}m{.2\textwidth - 2\tabcolsep}|
%		>{\centering\arraybackslash}m{.2\textwidth - 2\tabcolsep}|
%	}
%	\caption{Сравнение аналогов}\label{tbl:cmpAnalogues} \\\hline
%	 Критерии сравнения & Третьяковская галерея & Эрмитаж & Пушкинский музей & Лувр \\\hline    
%	 \endfirsthead
%	 \caption*{Продолжение таблицы~\ref{tbl:cmpAnalogues} } \\\hline
%	 Критерии сравнения & Третьяковская галерея & Эрмитаж & Пушкинский музей & Лувр \\\hline           
%	\endhead
%	\endfoot
%	
%	1 	& Название, автор, период создания, инвентарный номер, размер, материал, техника  
%		& Название, автор, период создания, инвентарный номер, размер, материал, техника  
%		& Название, автор, период создания, инвентарный номер, размер, материал, коллекция, страна, прежний владелец 
%		& Название, автор, период создания, инвентарный номер, размер, материал, техника, коллекция, страна, прежний владелец, местоположение \\\hline
%		
%	2 	& Название, автор, период 
%		& Нет возможности сортировки 
%		& Автор, период создания
%		& Название, автор, период создания, коллекция, актуальность, инвентарный номер \\\hline
%
%	3 	& Название, автор, категория, период, тип 
%		& Название, автор, период создания, инвентарный номер, размер, материал, техника  
%		& Название, автор, коллекция, страна, наличие изображения, прежний владелец
%		& Название, автор, период создания, коллекция, местонахождения \\\hline
%	4 & Нет & Нет & Только о присутствии/отсутствии в залах музея & Есть \\\hline
%	5 & - & + & + & -\\\hline
%	6 & + & + & + & +\\\hline
%	7 & + & + & + & +\\\hline
%%	5 & + & - & + & +\\\hline наличие выпадающего списка в фильтрах;	
%%	6 & + & + & + & +\\\hline возможность увидеть полную информацию о произведении искусства;
%%	9 & + & + & + & +\\\hline возможность спонсорства (программа “Друг музея”);
%	
%\end{longtable}
%
%\clearpage
%  
%%\imgw{\widthone\textwidth}{img/user-case-3}{Пользователи базы данных}
%\FloatBarrier
%%\textit{Рисунок -- Пользователи базы данных}
%
%\clearpage
%
%\section{Выбор базы данных по модели хранения}
%
%По модели хранения данных БД разделяют на 3 типа~\cite{introDBsys}:
%
%\begin{itemize}
%	\item \textbf{Дореляционные базы данных}. Данные формируются в виде структур, наиболее известными примерами которых являются инвертированные списки, деревья и графы. На основе этого определяется способ доступа к данным и работа с ними, в большинстве случаем используются указатели.
%%	представляют из себя ранние системы управления данными, включающие инвертированные списки (данные в файлах с индексами), иерархические (организация в виде дерева) и сетевые (представление в виде графа) модели. Они менее гибкие, сложнее в управлении и не поддерживали строгих ограничений целостности, в отличие от реляционных баз данных;
%%	
%	\item \textbf{Реляционные базы данных}. Данные формируются в виде взаимосвязанных таблиц (отношений), где каждая строка представляет запись, а столбцы -- атрибуты. Работа с данным возможна двумя способами: с использованием реляционной алгебры и реляционных исчислений. Накладываются четкие ограничения на формат представления данных в виде записей с фиксированным количеством атрибутов.
%%	организуют данные в виде таблиц (отношений), где каждая строка представляет запись, а столбцы — атрибуты. Такая модель состоит из трех основных компонент: структурной (описывает из каких объектов строится реляционная модель), целостной (определяет 2 базовых требования целостности -- целостность сущности и ссылочную целостность) и манипуляционной (описывает 2 эквивалентных способа манипулирования реляционными данными -- реляционную алгебру и реляционное исчисление);
%	\item \textbf{Постреляционные базы данных}. Данные представляются в виде объектов (таблиц), которые могут быть вложенными и содержать переменное количество полей. За счет этого возможно построение зависимостей одних данных от других и создание иерархии.
%	
%%	преодолевают ограничения реляционной модели. Они поддерживают гибкие структуры данных (документы, графы, ключ-значение), обеспечивают высокую производительность, масштабируемость и отказоустойчивость, подходят для больших объемов данных, неструктурированной информации и специализированных задач.;
%\end{itemize}
%
%На основе формализации поставленной задачи и в отсутствии необходимости поддержки гибких иерархичных структур данных была выбрана реляционная модель.
%
%%для реализации зависимостей выделенных элементов друг от друга и так как нет необходимости в поддержке гибких структуры данных, была выбрана реляционная модель.
%
%
%\section*{Вывод}
%В данном разделе были представлены: формализация задачи, анализ предметной области базы данных музея в виде информации на сайте Третьяковской галереи, Пушкинского музея, Эрмитажа и Лувра. Также была выбрана реляционная модель базы данных.
